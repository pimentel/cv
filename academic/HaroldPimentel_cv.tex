\documentclass[margin,line]{res}

\usepackage{hyperref}
\hypersetup{
  colorlinks=true,
  urlcolor=blue,
}

\oddsidemargin=-0.65in
\evensidemargin=-.5in
\textwidth=6.2in
\itemsep=0in
\parsep=0in
\topmargin -0.35in
\textheight 10.0in

\newenvironment{list1}{
  \begin{list}{\ding{113}}{%
      \setlength{\itemsep}{0in}
      \setlength{\parsep}{0in} \setlength{\parskip}{0in}
      \setlength{\topsep}{0in} \setlength{\partopsep}{0in}
      \setlength{\leftmargin}{0.17in}}}{\end{list}}
\newenvironment{list2}{
  \begin{list}{$\bullet$}{%
      \setlength{\itemsep}{0in}
      \setlength{\parsep}{0in} \setlength{\parskip}{0in}
      \setlength{\topsep}{0in} \setlength{\partopsep}{0in}
      \setlength{\leftmargin}{0.2in}}}{\end{list}}

\begin{document}

\name{Harold Pimentel \vspace*{.1in}}

\begin{resume}

\section{\sc Contact Information}
\vspace{.05in}

%#\begin{tabular}{@{}p{4in}p{2in}}
\begin{tabular}{r l}
  {\it Cell:}&  {(999) 999-9999} \\
  {\it E-mail:}&  {haroldpimentel@gmail.com} \\
  {\it Web:}& {\url{http://pimentel.github.io}} \\
  {\it GitHub:}& {\url{http://github.com/pimentel}} \\
  {\it Blog:}& {\url{http://haroldpimentel.wordpress.com}} \\
\end{tabular}

% \section{\sc Objective}
% To find an internship where I can apply my knowledge from statistics, machine
% learning, and computer science. In general, I am interested in analyzing
% high-dimensional data.

\section{\sc Education}
{\bf University of California, Berkeley}, Berkeley, California, USA

\vspace{-.3cm}
Ph.D. candidate, Computer Science \hfill {\bf
 Fall 2010 - Present}\\
\vspace{-.45cm}
\begin{list2}
\vspace*{1mm}
\item Advisor: Prof. Lior Pachter
\item Qualifying exam committee (May 2014): Doris Bachtrog, Haiyan Huang, Lior Pachter, Nir Yosef
\item Specialization: Computational biology, (bio)statistics, machine learning
\item Supported by NSF Graduate Research Fellowship Program (2011-2014)
\item Supported by GAANN fellowship, and Academic Excellence Award (2010-2011)
\end{list2}


M.A., Statistics \hfill {\bf
 December 2013}\\
\vspace{-.45cm}
\begin{list2}
\vspace*{1mm}
\item Advisor: Prof. Haiyan Huang
\item Specialization: Applied statistics
\item Thesis: Biclustering as an application of sparse canonical correlation analysis with RNA-Seq \\applications
\end{list2}



{\bf California State University, Fullerton}, Fullerton, California, USA

\vspace{-.3cm}
B.S., Computer Science \hfill {\bf August 2005 - May 2010}\\
\vspace{-.45cm}
\begin{list2}
\vspace*{1mm}
\item Specialization: Scientific computing
\item Minor: Mathematics
\item Graduated with summa cum laude
\end{list2}

\section{\sc Professional Experience}

{\bf 10X Genomics}, Pleasanton, California USA
\vspace{-.3cm}

{\em Computational biology intern} \hfill {\bf May 2014 - August 2014}\\
Using statistical and machine learning techniques (MART, GLM) to analyze
biological sequencing data from a novel sequencing platform. Also helped
develop analysis pipelines in Python and R.

{\bf University of California, Berkeley}, Berkeley, California USA
\vspace{-.3cm}

{\em Research Assistant} \hfill {\bf October 2010 - Present}\\
Professor Lior Pachter Lab. Generally focused on method development using
statistics and machine learning for mRNA-Seq technology. Currently working on
identifying incorrect gene annotations, differential expression coupled with
transcriptome abundance estimation, and analysis with biologist collaborators.
Past work: transcriptome mapping, visualizing fragment alignments. Also the
system administrator for the group.

{\em Research Assistant} \hfill {\bf June 2013 - Present}\\
Professor Haiyan Huang Lab. Developing methods for biclustering of linear
expression patterns by sparse canonical correlation analysis. Our method
generalizes many existing biclustering methods.

{\bf California State University, Fullerton (CSUF)}, Fullerton, California USA
\vspace{-.3cm}

{\em Research Assistant} \hfill {\bf December 2007 - May 2010}\\
Professor Spiros Courellis Lab. Shake\&wARP electron density map refinement
research project. Deployed a distributed and parallel computing environment used
for Phase Bias Removal in Macromolecular Crystallography. Developed new
techniques for alternative electron density maps, including new methods to
cluster maps based on similarity and returning a new averaged map based on the
most similar maps.

\newpage

{\bf Massachusetts Institute of Technology}, Cambridge, Massachusetts,
USA

\vspace{-.3cm}
{\em Summer Research Intern} \hfill {\bf June 2009 - August 2009}\\
Professor David Gifford Lab. Developed and implemented a technique for
determining the affinity of a transcription factor to a repetitive element
family in ChIP-seq reads. Application was used to test the affinity of Retinoic
Acid Receptor to ALU repetitive elements.

{\bf Thales Raytheon Systems}, Fullerton, California USA

\vspace{-.3cm}
{\em Information Solutions Intern} \hfill {\bf May 2007 - May 2008}\\
Provided reliable, innovative, effective, and efficient computing infrastructure
to employees. Tasks included support for PGP Encryption, server monitoring,
system administration, etc.


\section{\sc Publications}

% A. Conesa, P. Madrigal, M. Love, S. Tarazona, D. Gomez-Cabrero, A. Cervera,
% \underline{H. Pimentel}, A. McPherson, M. Szczesniak, D. Gaffney, L. Elo, X.
% Zhang, A. Mortazavi: ``Practical Guidelines and Standards in the Computational
% Analysis of RNA-seq Data", {\emph{Manuscript in preparation.}} *All authors
% contributed equally to this manuscript.
\underline{H. Pimentel}, N. Bray, P. Melsted, L. Pachter: ``Transcript level differential expression analysis with uncertainty estimates", {\emph{Manuscript in preparation.}}

\underline{H. Pimentel}, H. Huang: ``Generalized linear biclustering by sparse
canonical correlation analysis", {\emph{Manuscript in preparation.}}

\underline{H. Pimentel}, M. Parra, S. Gee, N. Mohandas, L. Pachter, J. G. Conboy: ``A dynamic intron retention program enriched in RNA processing genes regulates gene expression during terminal erythropoiesis", {\bf \emph{Nucleic Acids Research (2015)}}, doi: 10.1093/nar/gkv1168

N. Bray, \underline{H. Pimentel}, P. Melsted, L. Pachter: ``Near-optimal
RNA-Seq quantification", {\emph{arXiv preprint} (2015),
  http://arxiv.org/abs/1505.02710}

\underline{H. Pimentel}, J. G. Conboy, L. Pachter: ``Keep Me Around: Intron
Retention Detection and Analysis", {\emph{aXiv preprint} (2015), http://arxiv.org/abs/1510.00696}

\underline{H. Pimentel}, M. Parra, S. Gee, N. Mohandas, L. Pachter, J. G.
Conboy: ``An Erythroid-Specific Intron Retention Program Regulates Expression
of Selected Genes during Terminal Erythropoiesis", {\bf \emph{American Society
    of Hematology 2014}}, San Francisco, California, USA (December, 2014; Oral
Presentation). {\emph{Accompanying manuscript in preparation.}}

\underline{H. Pimentel}, M. Parra, S. Gee, D. Ghanem, X. An, J. Li, N.
Mohandas, L. Pachter, J. G. Conboy: ``A dynamic alternative splicing program
regulates gene expression during terminal erythropoiesis", {\bf \emph{Nucleic
    Acids Research (2014)}}, doi:10.1093/nar/gkt1388

D. Kim, G. Pertea, C. Trapnell, \underline{H. Pimentel}, R. Kelley, and S.
L. Salzberg, ``TopHat2: accurate alignment of transcriptomes in the presence of
insertions, deletions and gene fusions", {\bf \emph{Genome Biology
    14:R36 (2013)}}

C. Trapnell, A. Roberts, L. Goff, G. Pertea, D. Kim, D. R. Kelley,
\underline{H. Pimentel}, S. L. Salzberg, J. L. Rinn, L. Pachter: ``Differential
gene and transcript expression analysis of RNA-seq experiments with TopHat and
Cufflinks" {\bf \emph{Nature Protocols 7, 562–578 (2012)}}

A. Roberts, \underline{H. Pimentel}, C. Trapnell, L. Pachter: ``Identification
of novel transcripts in annotated genomes using RNA-Seq", {\bf
  \emph{Bioinformatics (2011) 27 (17): 2325-2329.}}

\newpage

\section{\sc Presentations}

\underline{H. Pimentel}, N. Bray, P. Melsted, L. Schaeffer, L. Pachter: ``Transcript level abundance estimation and differential expression with kallisto and sleuth", {\bf \emph{RNA-Seq symposium}}, Janssen Pharmaceuticals, WebEx (November, 2015)

\underline{H. Pimentel}, N. Bray, P. Melsted, L. Pachter: ``Transcript level differential analysis with sleuth", {\bf \emph{Genome Informatics}}, Cold Spring Harbor
Laboratory, USA (October, 2015; Poster Presentation).

\underline{H. Pimentel}, N. Bray, P. Melsted, L. Pachter: ``Transcript-level
quantification and differential expression with measurement uncertainty using
RNA-Seq", {\bf \emph{Ramsingh Lab}}, University of Southern California, USA (June,
2015; Ramsingh Group Presentation).

N. Bray, \underline{H. Pimentel}, P. Melsted, L. Pachter: ``Ultrafast accurate
RNA-Seq analysis", {\bf \emph{Biology of Genomes}}, Cold Spring Harbor
Laboratory, USA (June, 2015; Poster Presentation).

\underline{H. Pimentel}, H. Huang: ``Biclustering as an application of sparse
canonical correlation analysis", {\bf \emph{ISMB 2014}}, Boston, Massachusetts,
USA (July, 2014; Poster Presentation). Travel fellowship provided by FASEB
MARC.

\underline{H. Pimentel}, A. Roberts, C. Trapnell, L. Pachter: ``Visualizing
RNA-Seq fragment alignments", {\bf \emph{ISMB 2011}}, Vienna, Austria (July,
2011; Poster Presentation). Travel fellowship provided by FASEB MARC.

\underline{H. Pimentel}, S. H. Courellis, K. A. Kantardjieff, B. Rupp:
``Extending Shake\&wARP to Allow the Selection of the Refined Protein Electron
Density Map from a Set of Sub-optimal Solutions.'' {\bf \emph{CSU Program for
    Education and Research in Biotechnology 2010}}, Los Angeles, CA. (January
2010; Poster Presentation)

\underline{H. Pimentel}, S. H. Courellis, K. A. Kantardjieff, B. Rupp:
``Performance Evaluation of Shake\&wARP, a Cluster Enabled Application for
Macromolecular Crystallography.'' {\bf \emph{International Conference of
    Computing in Engineering, Science, and Information
    2009}}, Fullerton, CA. (April 2009; Oral Presentation and published in the
         IEEE Proceedings.)

\underline{H. Pimentel}, S. H. Courellis, K. A. Kantardjieff, B. Rupp:
``Hosting Multiple Simultaneous Applications over an Easy to
Administrate, Scalable, Extensible Platform Integrating Diverse
Computational Resources.'' {\bf \emph{Computer Science and Information
    Engineering 2009}}, Los Angeles,
CA. (April 2009; Poster Presentation and published in the
IEEE Proceedings vol.3, no., pp.67-71)

\underline{H. Pimentel}, S. H. Courellis, K. A. Kantardjieff, B. Rupp: ``A
performance evaluation of Shake\&wARP, an application used in macromolecular
crystallography for electron density map refinement.''  {\bf \emph{SACNAS
    National Conference 2009}}, Dallas, TX. (October 2009; Poster Presentation)

\underline{H. Pimentel}, S. Mahony, D. K. Gifford: ``Determining a
Transcription Factor's Affinity to Repetitive DNA.'' {\bf \emph{MIT
    Summer Research Program Poster Session 2009}}, MIT, Cambridge,
MA. (August 2009; Poster Presentation)

\underline{H. Pimentel}, S. H. Courellis, K. A. Kantardjieff, B. Rupp: ``A high
throughput implementation of Shake\&wARP, a Software Application used for Phase
Bias Removal in Macromolecular Crystallography.'' {\bf \emph{Natural Science and
    Mathematics Career Day 2009}}, CSUF, Fullerton, CA. (April 2009; Poster
Presentation)

\underline{H. Pimentel}, S. H. Courellis, K. A. Kantardjieff, B. Rupp:
``A high throughput implementation of Shake\&wARP, a Software
Application used for Phase Bias Removal in Macromolecular
Crystallography.'' {\bf \emph{CSU Program for Education and Research
   in Biotechnology 2009}}, Los Angeles, CA. (January 2009; Poster Presentation)

\newpage

\section{\sc Open source software}

\begin{list2}
  \item sleuth (2015) - Transcript level differential expression with uncertainty estimation (R)\\
    \url{https://github.com/pachterlab/sleuth}
  \item kallisto (2015) - Near optimal RNA-Seq quantification (C++11)\\
    \url{https://github.com/pachterlab/kallisto}
  \item Keep Me Around (2015) - Intron Retention Detection and Analysis (R) \\
    \url{https://github.com/pachterlab/kma}
  \item Fast SCCA (2014) - Implementation of SCCA using the NIPALS algorithm (C++, R) \\
    \url{https://github.com/pimentel/fscca}
  \item TopHat (2011) - Modules for transcriptome mapping since version 1.4.0 (C++) \\
    \url{http://ccb.jhu.edu/software/tophat}
  \item Other code can be found at \url{http://github.com/pimentel}
\end{list2}

\section{\sc Computer Skills}
\begin{list2}
\item Languages:
  \begin{list2}
    \item Proficient: R, Python
    \item Competent: C++, SQL
    \item Advanced beginner: Bash
  \end{list2}
\item Several other ``retired" languages which I haven't used in a while,
  but was once fluent in: csh, Java, Perl, MATLAB, Fortran
\item Unit testing
\item System administration in Linux/UNIX
\item Tools: Git, Vim, CMake, Make, Snakemake, Subversion, OpenPBS/Torque\\
\end{list2}
\vspace{-.65cm}

\section{\sc Honors and Awards}
FASEB MARC Travel Award for ISMB 2014 \hfill {\bf August
  July 2014}

National Science Foundation Graduate Research Fellowship Program (NSF GRFP)
\hfill {\bf August 2011 - May 2015}

FASEB MARC Travel Award for ISMB 2011 \hfill {\bf August
  July 2011}

Graduate Assistance in Areas of National Need (GAANN) Fellow \hfill {\bf August
  2010 - May 2011}

NIH Minority Access to Research Careers (MARC) Scholar \hfill {\bf
  June 2008 - May 2010}

Boeing Dean's Scholarship \hfill {\bf April 2009 - 2010}

``Outstanding poster presentation in Computer Science'' at SACNAS
Conference \hfill {\bf October 2009}

Southern California Auto Scholarship \hfill {\bf August 2008 - 2009}

\vspace*{1.5mm}
Outstanding Sophomore, Junior, and Senior at CSUF - Highest class GPA
\hfill {\bf May 2009}
\\
\hspace*{2cm} \hfill \textbf{2008} \\
\hspace*{2cm} \hfill \textbf{2007}

\vspace*{1.5mm}
Computer Science Dean's List at CSUF \hfill {\bf Fall 2005 - Spring 2010}

\section{\sc Professional Affiliations}
Member of Berkeley Science Network
(January 2013 - Present)

President for Association of Computing Machinery, CSUF Chapter
(August 2009 - May 2010)

% \vspace*{1.2mm}
President of the Omicron (CSUF) chapter for Upsilon Pi Epsilon,
international computer science honor society (January 2008 - May 2009)

% \vspace*{1.2mm}
Vice President for Association of Computing Machinery, CSUF Chapter
(January 2008 - May 2009)

% \vspace*{1.2mm}
Member of Orange County Linux Users Group as well as UNIX Users
Association of Southern California (June 2007 - May 2010)


% \vspace{0.15cm}
% For now, using the Dreyfus model of skill acquisition:
% http://en.wikipedia.org/wiki/Dreyfus_model_of_skill_acquisition
\newpage


\section{\sc Teaching}
GSI for ``Introduction to Statistical Computing" (STAT 243). Main course taught
by Chris Paciorek. (Fall 2015)

GSI for two sections of ``Structure and Interpretation of Computer Programs"
(CS 61A). Main course taught by Professor John DeNero. (Spring 2015)

\section{\sc Workshops}
``Algorithms for RNA-Seq" {\emph{RNA-Seq for the Next Generation}}. Hosted by Cold Spring Harbor Laboratory (2015). Slides posted shortly to \url{http://www.rnaseqforthenextgeneration.org/}

``eXpress transcript quantification" {\emph{*Seq I Workshop}}. Hosted by Laboratory of Mathematics and Computational biology (2012). \url{http://lmcb.wikispaces.com/eXpress+Walkthrough}

``Sequence alignment" {\emph{RNA-Seq workshop}}. Hosted by Computational Genomics Resource Laboratory (2011).  \url{http://cgrlucb.wikispaces.com/Sequence+alignment}

\section{\sc Outreach}
Presentations to biology and statistics classrooms at La Mirada High School on
computational biology (Spring 2014)

Mentor to summer research intern through SUPERB program at UC Berkeley (Summer
2011)

Mathematics, Engineering, Science Achievement (MESA) Robotics Workshop
(April, May 2009)
\vspace{-.25cm}

\section{\sc Hobbies}
Road and mountain cycling, bicycle mechanics, and hiking
\end{resume}
\end{document}
