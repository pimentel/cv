\documentclass[margin,line]{res}

\usepackage{hyperref}

\oddsidemargin=-0.65in
\evensidemargin=-.5in
\textwidth=6.2in
\itemsep=0in
\parsep=0in
\topmargin -0.35in
\textheight 10.0in

\newenvironment{list1}{
  \begin{list}{\ding{113}}{%
      \setlength{\itemsep}{0in}
      \setlength{\parsep}{0in} \setlength{\parskip}{0in}
      \setlength{\topsep}{0in} \setlength{\partopsep}{0in} 
      \setlength{\leftmargin}{0.17in}}}{\end{list}}
\newenvironment{list2}{
  \begin{list}{$\bullet$}{%
      \setlength{\itemsep}{0in}
      \setlength{\parsep}{0in} \setlength{\parskip}{0in}
      \setlength{\topsep}{0in} \setlength{\partopsep}{0in} 
      \setlength{\leftmargin}{0.2in}}}{\end{list}}


\begin{document}

\name{Harold Pimentel \vspace*{.1in}}

\begin{resume}
\section{\sc Contact Information}
\vspace{.05in}
\begin{tabular}{@{}p{4in}p{2in}}
 {\it Cell:}  (562) 448-2053 & \\             
 {\it E-mail:}  hpimentel@berkeley.edu & \\       
 {\it WWW:} \url{http://eecs.berkeley.edu/~pimentel} &
\end{tabular}

% \section{\sc Objective} 
% To find an internship where I can apply my knowledge from statistics, machine
% learning, and computer science. In general, I am interested in analyzing
% high-dimensional data.

\section{\sc Education}
{\bf University of California, Berkeley}, Berkeley, California, USA

\vspace{-.3cm}
Ph.D., Computer Science \hfill {\bf
 Fall 2010 - Present}\\
\vspace{-.45cm}
\begin{list2}
\vspace*{1mm}
\item Advisor: Prof. Lior Pachter
\item Specialization: Computational biology, (bio)statistics, machine learning
\item Supported by NSF Graduate Research Fellowship Program (2011-2014)
\item Supported by GAANN fellowship, and Academic Excellence Award (2010-2011)
\end{list2}


M.A., Statistics \hfill {\bf
 December 2013}\\
\vspace{-.45cm}
\begin{list2}
\vspace*{1mm}
\item Advisor: Prof. Haiyan Huang
\item Specialization: Applied statistics
\item Thesis: Biclustering as an application of sparse canonical correlation analysis with RNA-Seq \\applications
\end{list2}



{\bf California State University, Fullerton}, Fullerton, California, USA

\vspace{-.3cm}
B.S., Computer Science \hfill {\bf August 2005 - May 2010}\\
\vspace{-.45cm}
\begin{list2}
\vspace*{1mm}
\item Specialization: Scientific computing
\item Minor: Mathematics
\item Graduated with summa cum laude
\end{list2}

\section{\sc Professional Experience}

{\bf 10X Technologies}, Pleasanton, California USA
\vspace{-.3cm}

{\em Computational biology intern} \hfill {\bf May 2014 - August 2014}\\ 
Using statistics and machine learning techniques to analyze biological
sequencing data from a novel sequencing platform.

{\bf University of California, Berkeley}, Berkeley, California USA
\vspace{-.3cm}

{\em Research Assistant} \hfill {\bf October 2010 - Present}\\ 
Professor Lior Pachter Lab. Generally focused on method development using
statistics and machine learning for mRNA-Seq technology. Currently working on
identifying incorrect gene annotations, differential expression coupled with
transcriptome abundance estimation, and analysis with biologist collaborators. 
Past work: transcriptome mapping, visualizing fragment alignments.

{\em Research Assistant} \hfill {\bf June 2013 - Present}\\ 
Professor Haiyan Huang Lab. Developing methods for biclustering of linear
expression patterns. I am currently working with Prof. Huang to address some
regularization parameter estimation issues in our method, as well as working
with biologists to apply our method to modENCODE RNA-Seq data. In particular,
I've been working with the D. Melanogaster perturbation data set which looks at
85 different experimental conditions.


{\bf California State University, Fullerton (CSUF)}, Fullerton, California USA
\vspace{-.3cm}

{\em Research Assistant} \hfill {\bf December 2007 - May 2010}\\
Professor Spiros Courellis Lab. Shake\&wARP electron density map refinement
research project. Deployed a distributed and parallel computing environment used
for Phase Bias Removal in Macromolecular Crystallography. Developed new
techniques for alternative electron density maps, including new methods to
cluster maps based on similarity and returning a new averaged map based on the
most similar maps.

\newpage

{\bf Massachusetts Institute of Technology}, Cambridge, Massachusetts,
USA 

\vspace{-.3cm}
{\em Summer Research Intern} \hfill {\bf June 2009 - August 2009}\\
Professor David Gifford Lab. Developed and implemented a technique for
determining the affinity of a transcription factor to a repetitive element
family in ChIP-seq reads. Application was used to test the affinity of Retinoic
Acid Receptor to ALU repetitive elements.

{\bf Thales Raytheon Systems}, Fullerton, California USA

\vspace{-.3cm}
{\em Information Solutions Intern} \hfill {\bf May 2007 - May 2008}\\
Provided reliable, innovative, effective, and efficient computing infrastructure
to employees. Tasks included support for PGP Encryption, server monitoring,
system administration, etc.


\section{\sc Honors and Awards}
FASEB MARC Travel Award for ISMB 2014 \hfill {\bf August
  July 2014}

National Science Foundation Graduate Research Fellowship Program (NSF GRFP)
\hfill {\bf August 2011 - May 2015}

FASEB MARC Travel Award for ISMB 2011 \hfill {\bf August
  July 2011}

Graduate Assistance in Areas of National Need (GAANN) Fellow \hfill {\bf August
  2010 - May 2011}

NIH Minority Access to Research Careers (MARC) Scholar \hfill {\bf
  June 2008 - May 2010} 

Boeing Dean's Scholarship \hfill {\bf April 2009 - 2010}

``Outstanding poster presentation in Computer Science'' at SACNAS
Conference \hfill {\bf October 2009}

Southern California Auto Scholarship \hfill {\bf August 2008 - 2009}

\vspace*{1.5mm}
Outstanding Sophomore, Junior, and Senior at CSUF - Highest class GPA
\hfill {\bf May 2009}
\\
\hspace*{2cm} \hfill \textbf{2008} \\  
\hspace*{2cm} \hfill \textbf{2007}

\vspace*{1.5mm}
Computer Science Dean's List at CSUF \hfill {\bf Fall 2005 - Spring 2010}

\section{\sc Presentations and Publications}

\underline{H. Pimentel}, H. Huang: ``Biclustering as an application of sparse
canonical correlation analysis", {\bf \emph{ISMB 2014}}, Vienna, Austria (July,
2014; Poster Presentation). Travel fellowship provided by FASEB MARC. 

\underline{H. Pimentel}, L. Pachter: ``RNA-seq as a tool for analysing
differential expression", {\emph{In preparation.}}

\underline{H. Pimentel}, M. Parra, S. Gee, D. Ghanem, X. An, J. Li, N.
Mohandas, L. Pachter, J. G. Conboy: ``A dynamic alternative splicing program
regulates gene expression during terminal erythropoiesis", {\bf \emph{Nucleic
    Acids Research (2014)}}, doi:10.1093/nar/gkt1388 (Journal Publication)

D. Kim, G. Pertea, C. Trapnell, \underline{H. Pimentel}, R. Kelley, and S.
L. Salzberg, ``TopHat2: accurate alignment of transcriptomes in the presence of
insertions, deletions and gene fusions", {\bf \emph{Genome Biology
    14:R36 (2013)}} (Journal Publication)

C. Trapnell, A. Roberts, L. Goff, G. Pertea, D. Kim, D. R. Kelley,
\underline{H. Pimentel}, S. L. Salzberg, J. L. Rinn, L. Pachter: ``Differential
gene and transcript expression analysis of RNA-seq experiments with TopHat and
Cufflinks" {\bf \emph{Nature Protocols 7, 562–578 (2012)}} (Journal Publication)

A. Roberts, \underline{H. Pimentel}, C. Trapnell, L. Pachter: ``Identification
of novel transcripts in annotated genomes using RNA-Seq", {\bf
  \emph{Bioinformatics (2011) 27 (17): 2325-2329.}} (Journal Publication)

\underline{H. Pimentel}, A. Roberts, C. Trapnell, L. Pachter: ``Visualizing
RNA-Seq fragment alignments", {\bf \emph{ISMB 2011}}, Vienna, Austria (July,
2011; Poster Presentation). Travel fellowship provided by FASEB MARC.

\newpage

\underline{H. Pimentel}, S. H. Courellis, K. A. Kantardjieff, B. Rupp:
``Extending Shake\&wARP to Allow the Selection of the Refined Protein Electron
Density Map from a Set of Sub-optimal Solutions.'' {\bf \emph{CSU Program for
    Education and Research in Biotechnology 2010}}, Los Angeles, CA. (January
2010; Poster Presentation)

\underline{H. Pimentel}, S. H. Courellis, K. A. Kantardjieff, B. Rupp: ``A
performance evaluation of Shake\&wARP, an application used in macromolecular
crystallography for electron density map refinement.''  {\bf \emph{SACNAS
    National Conference 2009}}, Dallas, TX. (October 2009; Poster Presentation)

\underline{H. Pimentel}, S. Mahony, D. K. Gifford: ``Determining a
Transcription Factor's Affinity to Repetitive DNA.'' {\bf \emph{MIT
    Summer Research Program Poster Session 2009}}, MIT, Cambridge,
MA. (August 2009; Poster Presentation)

\underline{H. Pimentel}, S. H. Courellis, K. A. Kantardjieff, B. Rupp: ``A high
throughput implementation of Shake\&wARP, a Software Application used for Phase
Bias Removal in Macromolecular Crystallography.'' {\bf \emph{Natural Science and
    Mathematics Career Day 2009}}, CSUF, Fullerton, CA. (April 2009; Poster
Presentation)

\underline{H. Pimentel}, S. H. Courellis, K. A. Kantardjieff, B. Rupp:
``Performance Evaluation of Shake\&wARP, a Cluster Enabled Application for
Macromolecular Crystallography.'' {\bf \emph{International Conference of
    Computing in Engineering, Science, and Information
    2009}}, Fullerton, CA. (April 2009; Oral Presentation and published in the
         IEEE Proceedings.)

\underline{H. Pimentel}, S. H. Courellis, K. A. Kantardjieff, B. Rupp:
``Hosting Multiple Simultaneous Applications over an Easy to
Administrate, Scalable, Extensible Platform Integrating Diverse
Computational Resources.'' {\bf \emph{Computer Science and Information
    Engineering 2009}}, Los Angeles,
CA. (April 2009; Poster Presentation and published in the
IEEE Proceedings vol.3, no., pp.67-71)


\underline{H. Pimentel}, S. H. Courellis, K. A. Kantardjieff, B. Rupp:
``A high throughput implementation of Shake\&wARP, a Software
Application used for Phase Bias Removal in Macromolecular
Crystallography.'' {\bf \emph{CSU Program for Education and Research
   in Biotechnology 2009}}, Los Angeles, CA. (January 2009; Poster Presentation)


\section{\sc Professional Affiliations}
Member of Berkeley Science Network
(January 2013 - Present) 

President for Association of Computing Machinery, CSUF Chapter
(August 2009 - May 2010) 

% \vspace*{1.2mm}
President of the Omicron (CSUF) chapter for Upsilon Pi Epsilon,
international computer science honor society (January 2008 - May 2009) 

% \vspace*{1.2mm}
Vice President for Association of Computing Machinery, CSUF Chapter
(January 2008 - May 2009) 

% \vspace*{1.2mm}
Member of Orange County Linux Users Group as well as UNIX Users
Association of Southern California (June 2007 - May 2010)

\section{\sc Open source software}
Wrote numerous modules in TopHat spliced-alignment software to support transcriptome mapping since version 1.4.0. (C++). http://tophat.cbcb.umd.edu/. Other projects available on my Github page: http://github.com/pimentel/.
\newpage
% \vspace{0.15cm}
\section{\sc Computer Skills} 
\begin{list2}
\item Languages:  C++, Python, SQL, Perl, Java, bash, csh, and others
\item Libraries: Boost, STL, Java Swing
\item Mathematical Languages: R, MATLAB
\item System administration in Linux/UNIX
\item Tools: Git, Subversion, CMake, Ant, VIM, OpenPBS/Torque\\ 
\end{list2}
\vspace{-.65cm}
\section{\sc Outreach}
Presentations to biology and statistics classrooms at La Mirada High School on computational biology (Spring 2014)

Mentor to summer research intern through SUPERB program at UC Berkeley (Summer 2011)


Mathematics, Engineering, Science Achievement (MESA) Robotics Workshop
(April, May 2009)
\vspace{-.25cm}

\section{\sc Hobbies}
Road and mountain cycling, bicycle mechanics, and hiking
\end{resume}
\end{document}
