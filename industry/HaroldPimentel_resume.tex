\documentclass[margin,line]{res}

\usepackage{hyperref}
\hypersetup{
  colorlinks=true,
  urlcolor=blue,
}

\oddsidemargin=-0.65in
\evensidemargin=-.65in
\textwidth=6.2in
\itemsep=0in
\parsep=0in
\topmargin -0.35in
\textheight 10.0in

\newenvironment{list1}{
  \begin{list}{\ding{113}}{%
      \setlength{\itemsep}{0in}
      \setlength{\parsep}{0in} \setlength{\parskip}{0in}
      \setlength{\topsep}{0in} \setlength{\partopsep}{0in}
      \setlength{\leftmargin}{0.17in}}}{\end{list}}
\newenvironment{list2}{
  \begin{list}{$\bullet$}{%
      \setlength{\itemsep}{0in}
      \setlength{\parsep}{0in} \setlength{\parskip}{0in}
      \setlength{\topsep}{0in} \setlength{\partopsep}{0in}
      \setlength{\leftmargin}{0.2in}}}{\end{list}}


\begin{document}

\name{Harold Pimentel \vspace*{.1in}}

\begin{resume}
\section{\sc Contact Information}
\vspace{.05in}
\begin{tabular}{@{}p{4in}p{2in}}
 {\it Cell:}  (999) 999-9999 & \\
 {\it E-mail:}  haroldpimentel@gmail.com & \\
 {\it WWW:} \url{http://cs.berkeley.edu/~pimentel} &
\end{tabular}

\section{\sc Objective}
To find an internship where I can apply my statistics, machine learning, and
computer science skills. In general, I am interested in data science, especially
analyzing high-dimensional data.

\section{\sc Education}
{\bf University of California, Berkeley}, Berkeley, California, USA

\vspace{-.3cm}
Ph.D. candidate, Computer Science \hfill {\bf
  Fall 2010 -- Fall 2015 (expected) }\\
\vspace{-.45cm}
\begin{list2}
  \vspace*{1mm}
\item Advisor: Prof. Lior Pachter
\item Specialization: Computational biology, (bio)statistics, machine learning
\item Supported by NSF Graduate Research Fellowship Program (2011--2015)
\item Supported by GAANN fellowship, and Academic Excellence Award (2010--2011)
\end{list2}

M.A., Statistics \hfill {\bf
 December 2013}\\
\vspace{-.45cm}
\begin{list2}
\vspace*{1mm}
\item Advisor: Prof. Haiyan Huang
\item Specialization: Applied statistics
\item Thesis: Biclustering as an application of sparse canonical correlation analysis with RNA-Seq \\applications
\end{list2}

{\bf California State University, Fullerton}, Fullerton, California, USA

\vspace{-.3cm}
B.S., Computer Science \hfill {\bf August 2005 -- May 2010}\\
\vspace{-.45cm}
\begin{list2}
\vspace*{1mm}
\item Specialization: Scientific computing
\item Minor: Mathematics
\item Graduated with summa cum laude
\end{list2}

% \vspace{-0.3cm}
% \section{\sc Relevant Graduate Coursework}
% Statistical Learning Theory (CS 281A), Theoretical Statistics I (STAT 210A), Statistical Models: Theory and Application (STAT 215B), Advanced Computer Systems (CS 262A)
% \vspace{0.1cm}

\section{\sc Professional Experience}

{\bf 10X Technologies}, Pleasanton, California USA
\vspace{-.3cm}

{\em Computational Biology Intern} \hfill {\bf May 2014 -- August 2014}\\
Using statistics and machine learning techniques (MART, GLM) to analyze
biological sequencing data from a novel sequencing platform. Also helped
developed analysis pipelines in Python and R.

{\bf University of California, Berkeley}, Berkeley, California USA
\vspace{-.3cm}

{\em Research Assistant} \hfill {\bf October 2010 -- Present}\\
Professor Lior Pachter Lab. Development of data analysis methods using
statistics and machine learning for mRNA-Seq data. Currently working on
identifying incorrect gene annotations using linear programming, differential
expression coupled with transcriptome abundance estimation using shrinkage
estimators, and data analysis with biologist collaborators. Past work:
transcriptome mapping algorithms, visualizing fragment alignments. Also the
system administrator for the group.

{\em Research Assistant} \hfill {\bf June 2013 -- Present}\\
Professor Haiyan Huang Lab. Developing methods for biclustering of linear
expression patterns by extending sparse canonical correlation analysis. Our
method generalizes many existing biclustering methods. We are focusing on
applications to large heterogeneous gene expression experiments and time series
gene expression experiments.

{\bf California State University, Fullerton}, Fullerton, California USA
\vspace{-.3cm}

{\em Research Assistant} \hfill {\bf December 2007 -- May 2010}\\
Professor Spiros Courellis Lab. Deployed a distributed and parallel computing
environment used for Phase Bias Removal in Macromolecular Crystallography.
Developed new methods to cluster maps based on similarity and returning a new
averaged map based on the most similar maps.

\newpage

{\bf Massachusetts Institute of Technology}, Cambridge, Massachusetts,
USA

\vspace{-.3cm}
{\em Summer Research Intern} \hfill {\bf June 2009 -- August 2009}\\
Professor David Gifford Lab. Developed and implemented a technique for
determining the affinity of a transcription factor to a repetitive element
family in ChIP-seq reads. Application was used to test the affinity of Retinoic
Acid Receptor to ALU repetitive elements.

\section{\sc Open source software}
\begin{list2}
  \item Fast SCCA (2014) - Implementation of SCCA using the NIPALS algorithm (C++, R) \\
    \url{http://github.com/pimentel/fscca}
  \item TopHat (2011) - Modules for transcriptome mapping since version 1.4.0 (C++) \\
    \url{http://ccb.jhu.edu/software/tophat}
  \item Miscellaneous code can be found at \url{http://github.com/pimentel}
\end{list2}

% \vspace{0.15cm}
% For now, using the Dreyfus model of skill acquisition:
% http://en.wikipedia.org/wiki/Dreyfus_model_of_skill_acquisition
\section{\sc Computational Skills}
\begin{list2}
\item Languages:
  \begin{list2}
    \item Proficient: R, Python
    \item Competent: C++, SQL
    \item Advanced beginner: Bash
    \item Experience with but currently rarely use: Java, Perl, MATLAB, Fortran, csh
  \end{list2}
\item Unit testing
\item System administration in Linux/UNIX
\item Tools: Git, CMake, Vim, OpenPBS/Torque\\
\end{list2}
\vspace{-.65cm}

\section{\sc Selected Presentations and Publications (since 2010)}

\underline{H. Pimentel}, M. Parra, S. Gee, N. Mohandas, L. Pachter, J. G.
Conboy: ``An Erythroid-Specific Intron Retention Program Regulates Expression
of Selected Genes during Terminal Erythropoiesis", {\bf \emph{American Society
    of Hematology 2014}}, San Francisco, California, USA (December, 2014; Oral
Presentation).

\underline{H. Pimentel}, H. Huang: ``Biclustering as an application of sparse
canonical correlation analysis", {\bf \emph{ISMB 2014}}, Vienna, Austria (July,
2014; Poster Presentation). Travel fellowship provided by FASEB MARC.

\underline{H. Pimentel}, M. Parra, S. Gee, D. Ghanem, X. An, J. Li, N.
Mohandas, L. Pachter, J. G. Conboy: ``A dynamic alternative splicing program
regulates gene expression during terminal erythropoiesis", {\bf \emph{Nucleic
    Acids Research (2014)}}, doi:10.1093/nar/gkt1388 (Journal Publication)

D. Kim, G. Pertea, C. Trapnell, \underline{H. Pimentel}, R. Kelley, and S.
L. Salzberg, ``TopHat2: accurate alignment of transcriptomes in the presence of
insertions, deletions and gene fusions", {\bf \emph{Genome Biology
    14:R36 (2013)}} (Journal Publication)

A. Roberts, \underline{H. Pimentel}, C. Trapnell, L. Pachter: ``Identification
of novel transcripts in annotated genomes using RNA-Seq", {\bf
  \emph{Bioinformatics (2011) 27 (17): 2325-2329.}} (Journal Publication)

\underline{H. Pimentel}, A. Roberts, C. Trapnell, L. Pachter: ``Visualizing
RNA-Seq fragment alignments", {\bf \emph{ISMB 2011}}, Vienna, Austria (July,
2011; Poster Presentation). Travel fellowship provided by FASEB MARC.

\section{\sc Honors and Awards \\ (Since 2010)}
FASEB MARC Travel Award for ISMB 2014 \hfill {\bf July 2014}

National Science Foundation Graduate Research Fellowship \hfill {\bf August 2011 -- May 2015} \\
Program (NSF GRFP)

FASEB MARC Travel Award for ISMB 2011 \hfill {\bf July 2011}

Graduate Assistance in Areas of National Need (GAANN) Fellow \hfill {\bf August
  2010 -- May 2011}


\section{\sc Outreach \\ (Since 2010)}
Presentations to biology and statistics classrooms at La Mirada High School on
computational biology (Spring 2014)

Mentor to summer research intern through SUPERB program at UC Berkeley (Summer
2011)

\vspace{-.25cm}

\section{\sc Hobbies}
Road and mountain cycling, bicycle mechanics, and hiking
\end{resume}
\end{document}
